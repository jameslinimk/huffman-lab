\documentclass[10pt]{article}

% DO NOT EDIT THE LINES BETWEEN THE TWO LONG HORIZONTAL LINES

%---------------------------------------------------------------------------------------------------------

% Packages add extra functionality.
\usepackage{times,graphicx,epstopdf,fancyhdr,amsfonts,amsthm,amsmath,algorithm,algorithmic,xspace,hyperref}
\usepackage[left=1in,top=1in,right=1in,bottom=1in]{geometry}
\usepackage{sect sty}	%For centering section headings
\usepackage{enumerate}	%Allows more labeling options for enumerate environments 
\usepackage{epsfig}
\usepackage[space]{grffile}
\usepackage{amsmath}
\usepackage{listings}
\usepackage{hyperref}
\hypersetup{
    colorlinks=true,
    linkcolor=blue,
    filecolor=magenta,      
    urlcolor=cyan,
}

% This will set LaTeX to look for figures in the same directory as the .tex file
\graphicspath{.} %The dot means current directory.

\pagestyle{fancy}

\lhead{\YourName}
\chead{Lab: Huffman Coding}
\rhead{\today}
\lfoot{Theoretical Computer Science}
\cfoot{\thepage}
\rfoot{2021-2022}

% Some commands for changing header and footer format
\renewcommand{\headrulewidth}{0.4pt}
\renewcommand{\headwidth}{\textwidth}
\renewcommand{\footrulewidth}{0.4pt}

% These let you use common environments
\newtheorem{claim}{Claim}
\newtheorem{definition}{Definition}
\newtheorem{theorem}{Theorem}
\newtheorem{lemma}{Lemma}
\newtheorem{observation}{Observation}
\newtheorem{task}{Task}
\newtheorem{ctask}{Challenge Task}
\newtheorem{ov}{Overview}
\newtheorem{question}{Question}

\setlength{\parindent}{0cm}


%---------------------------------------------------------------------------------------------------------

% DON'T CHANGE ANYTHING ABOVE HERE

% Edit below as instructed
\newcommand{\YourName}{}	% Put your name in the braces
\newcommand{\PSNumber}{1}			% Put the problem set # in the braces
\newcommand{\ProblemHeader}{Problem \ProblemNumber}	% Don't change this

\begin{document}

\textbf{Overview:}
For this assignment, you will be creating a Huffman encoding of part of the ASCII alphabet based on the frequencies of characters in a book of your choice from Project Gutenberg. You will compare the space efficiency of that encoding to the space efficiency of a fixed-length (worst-case-optimal) encoding.


\vspace{\baselineskip}	% Add some vertical space

\begin{task}
Pick a book from \href{https://www.gutenberg.org/}{Project Gutenberg} and download it as a Plaintext file (.txt). Create a project for your lab that includes the text file as a resource and all necessary program files (Java classes, Scala classes, Python scripts, etc). 
\end{task}

\begin{task}
Write a function countFrequencies() that reads through your .txt file and creates a Map (Java: HashMap; Python: dictionary; Scala: Map) relating each distinct ASCII character that you find to the number of times it occurs in your .txt. Print that Map to ensure that your function works.
\end{task}

\begin{task}
Create a TreeVertex class with the following fields:
\begin{itemize}
    \item char c
    \item int weight
    \item TreeVertex leftChild
    \item TreeVertex rightChild
    \item int height (optional)
\end{itemize}
\end{task}

\vspace{\baselineskip}
\begin{task}
Write a function called buildTree(freqs) that constructs a Huffman tree for encoding the characters in your source file based on your frequency Map [freqs]. buildTree should return a TreeVertex object that is the root of your tree.

\vspace{\baselineskip}
Use a PriorityQueue in your function. PriorityQueues are Queues that automatically dequeue the element with highest 'priority'. In this case, you will want a PriorityQueue containing TreeVertex instances, in which the prioritized vertices are those with lowest weight (representing least frequent characters).

\textbf{PriorityQueue Documentation}:
\begin{itemize}
    \item \href{https://docs.python.org/3/library/queue.html}{Python}
    \item \href{https://docs.oracle.com/javase/7/docs/api/java/util/PriorityQueue.html}{Java}
    \item \href{https://www.scala-lang.org/api/2.12.6/scala/collection/mutable/PriorityQueue.html}{Scala}
\end{itemize}
\end{task}



\vspace{\baselineskip}
\begin{task}
Write a function called createCodeMap(root) that uses your tree to construct a Map relating each character in your tree to a unique bitstring to encode that character. createCodeMap should return that Map object.
\end{task}

{\large Answer the following questions in comments at the top of your code file.}

\vspace{\baselineskip}
\vspace{\baselineskip}
\begin{question}
How many bits are needed to encode your entire .txt file using your Huffman code?
\end{question}

\begin{itemize}
	\item It took 5,769,341 bits to encode
\end{itemize}

\vspace{\baselineskip}
\begin{question}
If we were making our own fixed bit-length encoding of your .txt file, what would be the minimum necessary bit-length for encoding a character? Using this fixed-length encoding, how many bits would be necessary to encode the entire .txt file? Comparing this to your answer from Question 1, how much space do you save by using the Huffman code rather than the fixed-length encoding?
\end{question}

\begin{itemize}
	\item It would take 7 bits to encode a character and 8,823,605 bits to encode with fixed length
	\item The Huffman code saved 34.61\% (3,054,264 bits)
\end{itemize}

\vspace{\baselineskip}
\vspace{\baselineskip}

{\large Below this line, all additional problems are extensions/challenge problems, which are optional.}


\vspace{\baselineskip}
\begin{ctask}
Find a list of average frequencies of ASCII characters and use this to create an 'average case' encoding. Generate a table of average frequencies and generate a Huffman tree and encoding Map. If you use this encoding, how does its efficiency compare to the efficiency of your Huffman Code custom-built for this .txt file, and how does it compare to the efficiency of the fixed-length encoding?
\end{ctask}

\begin{itemize}
	\item Used \href{https://github.com/piersy/ascii-char-frequency-english}{ascii character frequencies in english} for average frequencies
	\item Average Huffman took 6,656,750 bits to encode
	\item Custom Huffman 13.33\% (887,409 bits) more efficient than average
	\item Average Huffman saved 24.56\% (2,166,855 bits) from fixed length
\end{itemize}

\vspace{\baselineskip}
\begin{ctask}
Find another book on \href{https://www.gutenberg.org/}{Project Gutenberg}, read in the frequencies, and calculate the number of bits required to encode this new book with your Huffman code that you generated in Task 5.

Compare the space efficiency of encoding this new book using the Huffman Code from Task 4 with the space efficiency of encoding this new book using the Huffman Code from Challenge Task 1. Which is more efficient, and by how much? Why do you think that is?

\end{ctask}

\begin{itemize}
	\item Takes 238,952 bits to encode using Task 5 Huffman
	\item Takes 212,383 bits to encode using average Huffman
	\item Takes 312,872 bits to encode using smallest fixed length
	\item Task 5 Huffman is 11.12\% (26,569 bits) more efficient
	\item I believe that task 5 Huffman is slightly more efficient because of the chunk of the text at the top of the file ("This ebook is for the use of anyone anywhere in the United States...") is shared between the two books, and the custom Huffman is optimized for project gutenberg books. But, the difference is small enough that it could be due to random chance.
\end{itemize}

\vspace{\baselineskip}

\vspace{\baselineskip}
I found this problem set to be (underline one):
\begin{enumerate}
\item Easy as pie
\item Moderately easy
\item \underline{About what I would expect on average}
\item Challenging but fair
\item Are you kidding me?  This is supposed to be 400-level half credit!!!
\end{enumerate}

\textbf{Comments:} Code is on \href{https://github.com/jameslinimk/huffman-lab}{Github}. I also answered the questions here for convenience.

%End of feedback section

% DO NOT DELETE ANYTHING BELOW THIS LINE
\end{document}
